\documentclass[a4paper,10pt]{article}

\usepackage{tabularx}
\usepackage{array}
\usepackage{enumitem}

\usepackage{pifont}
\usepackage[scaled]{helvet}
\usepackage[T1]{fontenc}
\usepackage{utopia}
\usepackage{multirow}
\usepackage{enumitem}

%Graphics - Colors
\usepackage[dvipsnames, table]{xcolor}
\usepackage{graphicx}
\usepackage{fullpage}
\usepackage[top=0.5in, bottom=1in, left=0.5in, right=0.50in]{geometry}

\usepackage[most]{tcolorbox}
\addtolength{\textheight}{1.85cm}

%Setup hyperref package, and colours for links
\usepackage{hyperref}
\usepackage{url,parskip}
\definecolor{linkcolour}{rgb}{0,0.2,0.6}
\hypersetup{colorlinks,breaklinks, urlcolor=linkcolour, linkcolor=linkcolour}

\usepackage{titlesec} %custom section
\titlespacing{\section}{0pt}{12pt}{0pt}
% title bg color
\definecolor{bgfill}{rgb}{0.90, 0.935, 0.95}

\newcommand{\colorsection}[1]{%
    \colorbox{bgfill}{\parbox{\dimexpr\textwidth-2\fboxsep}{\ #1}}}
\newcommand{\colorsectionnonumber}[1]{%
    \colorbox{bgfill}{\parbox{\dimexpr\textwidth-2\fboxsep}{#1}}}
\titleformat{\section}
  {\normalfont\LARGE\bfseries\raggedright} % Format of the title text
  {}        % Label (e.g., section numbers) — empty means no number
  {0em}     % Horizontal separation between label and title — 0 since no label
  {\colorsection}  % Code to execute before the title — likely sets a color
  %[\titlerule]     % Code to execute after the title — adds a horizontal rule


% BIG OH
\DeclareRobustCommand{\bigO}{%
  \text{\usefont{OMS}{cmsy}{m}{n}O}%
}

% Job title macro
\newcolumntype{M}{>{\hsize=.6\hsize}X}
\newcolumntype{S}{>{\hsize=.3\hsize}X}
\definecolor{darkmahogany}{rgb}{0.6, 0.25, 0.0}
\newcommand*{\mahogany}{\textcolor{darkmahogany}}
\newcommand{\jobtitle}[4]{
\vspace{0.3cm}
\begin{tabularx}
{\textwidth}
{ >{\raggedright\arraybackslash}p{6.5cm}
  >{\raggedright\arraybackslash}X
  >{\raggedleft\arraybackslash}p{4cm} }
\fontsize{13pt}{15pt}\selectfont\textbf{#1} &

% \includegraphics[height=12pt,width=12pt]{#2}
\large\textbf{\mahogany{#2}} \normalsize\textbf{(#3)} &
\textit{\large\textbf{#4}}
\end{tabularx}
\vspace{-0.4cm}
}

\begin{document}
\pagestyle{empty} % non-numbered pages

%--------------------TITLE-------------
\begin{minipage}{.40\linewidth}
\begin{flushleft}
\Huge \textsc{Pavlos Rontidis}
\end{flushleft}
\end{minipage}
\hfill
\begin{minipage}{.25\linewidth}
\begin{flushright}
\href{tel:15555555555}{+1 718 974 0661} \\
\href{mailto:pavlos.rontidis@pm.me}{pavlos.rontidis@pm.me}\\
\end{flushright}
\end{minipage}
\hfill
\begin{minipage}{.30\linewidth}
\begin{flushright}
\href{https://github.com/pront}{Github Profile} \\
Last update: \today
\end{flushright}
\end{minipage}

\noindent\makebox[\linewidth]{\rule{\linewidth}{0.4pt}}

\section{Employment}
% Datadog
\jobtitle{Tech Lead, SWE}{Datadog}{New York, NY}{Jun 2023 – Present}
\begin{itemize}[leftmargin=.27in,label=\ding{226}] \setlength\itemsep{-0.1cm}
    \item \textbf{Community Open Source Engineering}

    Founding member of Datadog’s Community Open Source Engineering team, where I lead the development of \href{https://github.com/vectordotdev/vector}{Vector}, a widely adopted observability data platform written in \textsc{Rust}. The project was recently \href{https://news.ycombinator.com/item?id=39737122}{featured on Hacker News}. I oversee all facets of the project, including defining the roadmap, driving feature development, reviewing pull requests, maintaining/optimizing CI workflows, and shaping the interview process for growing the team.

    In addition, I contribute to the \href{https://github.com/vectordotdev/vrl}{Vector Remap Language (VRL)}, where I introduced new compiler features like unused expression detection. I am currently leading this project by managing releases and providing guidance for community contributions.

    In early 2024, I was part of the core team that rebooted the Observability Pipelines (OP) product — based on vector.dev — helping redefine its architecture and long-term direction. Following that, I led several high-impact OP projects, including the design and implementation of a live capture system for ingesting real-time events from a distributed worker fleet.
\end{itemize}

% Dropbox
\jobtitle{Senior Software Engineer}{Dropbox}{New York, Remote}{Aug 2021 - Jun 2023}
\begin{itemize}[leftmargin=.27in,label=\ding{226}] \setlength\itemsep{-0.1cm}
	\item \textbf{Core Sync} - \textit{Synchronization Engine}

    The synchronization engine is at the heart of most Dropbox products. As a member of the Core Sync team, I led several projects in this area,
    specifically on replacing our Linux kernel extension using Apple's FileProvider framework.
    The overarching project was a multi-year, mission-critical effort,
    and this new Dropbox engine was rolled out to millions of users while I was there.
    Refer to \href{https://dropbox.tech/infrastructure/rewriting-the-heart-of-our-sync-engine}{this article}
    for more details on how the engine was rewritten using \textsc{Rust}.
\end{itemize}

% Google
\jobtitle{Software Engineer}{Google}{New York, Remote}{Jun 2020 - Aug 2021}
\begin{itemize}[leftmargin=.27in,label=\ding{226}] \setlength\itemsep{-0.1cm}
	\item \textbf{Unified Traffic Engineering} - \textit{Distributed Systems, Middleware, Networking Infrastructure}

The team owned all micro-services responsible for programming of proprietary as well as commercial networking hardware that serves the \textbf{immense} amount of internet traffic passing through the Google data centers.

I spearheaded the design and implementation of a real-time monitoring system aimed at validating consistency between specialized network databases across multiple domains. This necessitated communication and coordination with several teams and stakeholders. Used a plethora of proprietary technologies. After delivering the project, I felt it was time to move on to \textsc{Rust} and other more modern technologies.
\end{itemize}

%Bloomberg
\jobtitle{Senior Software Engineer}{Bloomberg LP}{New York, NY}{Nov 2014 - Jul 2020}
\begin{itemize}[leftmargin=.27in,label=\ding{226}] \setlength\itemsep{-0.1cm}
    %Gateways
	\item \textbf{Gateway Architecture} - \textit{Distributed Systems, Middleware, Networking Infrastructure}

I was involved in the modernization of legacy \textit{Bloomberg gateways}
and the implementation of low-level network protocols in \textsc{C\texttt{++}}.
This project was a major refactoring challenge,
as all Bloomberg terminals connected to these gateways, and a migration without disrupting users was a key requirement.


	%DMP
    \vspace{0.2cm}
    \item \textbf{Publish/Subscribe middleware} - \textit{Distributed Systems}

I worked in the Message Oriented Middleware team, specifically on DMP, which is a proprietary publish/subscribe middleware that runs on multiple platforms. Almost every application on Bloomberg uses this system to this day.
The open sourced \href{https://bloomberg.github.io/blazingmq/}{blazingmq} is a great insight into what the team was working on and it shares quite a few modules with DMP.

I led several DMP projects throughout my tenure, taking responsibility for all stages of development, from designing new features to deployment and user support.
I designed and implemented a low-level thread-safe \textit{subscription API} in \textsc{C\texttt{++}}.
Additionally, I designed and implemented an application for subscription configuration and routing validation using \textsc{C\texttt{++}} for the backend and JavaScript for the admin application.
I also revamped various side projects related to this framework to support power users.

    \vspace{0.2cm}
    \item \textbf{Real-time Data Processing} - \textit{London office 2014-2017}

This position kickstarted my career in Software Engineering.
My job responsibilities involved maintaining, extending, and optimizing \textit{DataLayer},
a proprietary in-memory data processing framework that powers Bloomberg's financial applications. \textsc{(C\texttt{++})}

    Designed and implemented \textit{profiler} and \textit{replay debugger} for the aforementioned framework. \textsc{(C++, JavaScript)}

    JavaScript \textit{transpiler} for module inlining and static analysis of Datalayer modules. \textsc{(JavaScript)}

    Designed \textit{HBase} database for historical financial data and implemented map-reduce Hadoop jobs. \textsc{(Java)}
\end{itemize}

% **** EDUCATION SECTION ****
\section{Education}
\jobtitle{MSc in Software Engineering}{Imperial College London}{UK}{Sep 2013 - Sep 2014}
\begin{itemize}[leftmargin=.27in,label=\ding{226}] \setlength\itemsep{0cm}
\vspace{0.2cm}
\item\textbf{Greade:} Merit
\item \normalsize \textbf{Thesis}:
\textit{Parallel Architecture for Large-scale Spiking Neural Network Simulation}
\vspace{-0.1cm}

Implemented an Open MPI layer on top of NeMo (spiking neural network simulator) to enable the usage of multiple CPUs/GPUs in an effort to aid the neuroscience community.  The main goal was to support larger networks and heterogeneous devices. The project was awarded the \textit{distinction} grade.

\end{itemize}

\vspace{0.1cm}
\jobtitle{Engineer's Degree}{Technical University Crete}{Greece}{Sep 2007 - Jul 2013}
\vspace{0.2cm}

\begin{itemize}[leftmargin=.27in,label=\ding{226}]\setlength\itemsep{-0.1cm}
\item \textbf{Title:} \href{https://en.wikipedia.org/wiki/Engineer's_degree#Europe}{Engineer's degree} in Electronic and Computer Engineering

\item\textbf{GPA:} 8.6/10 (Excellent), Top 7\% since 1990

\item \textbf{Thesis:}
\textit{Model Driven Development in Wireless Sensor Networks}
\vspace{-0.1cm}

Designed a Domain Specific Language (DSL) and implemented a model compiler which generated NesC code from the DSL code. Contributed to WSN-DPCM, a project that spans across several technical universities and companies, funded by ARTEMIS.

\item \textbf{Awards:} Multiple scholarships, including the Limmat Stiftung Excellence Awards

\end{itemize}

% **** OPEN SOURCE CONTRIB SECTION ****
\section{Open Source Contributions}
\vspace{0.25cm}
\begin{itemize}[leftmargin=.27in,label=\ding{226}]
\setlength\itemsep{0.1cm}
\item Implemented a well known Gossip protocol algorithm, known as Randomized Rumor Spreading, and integrated it into a very well-known open source project written in \textsc{(C\texttt{++})}. This replaced a pre-existing naive $\bigO(n^2)$ network broadcasting mechanism with a more efficient mechanism requiring only $\bigO(n\ln n\ln n)$ transmissions.
\end{itemize}

\end{document}
